% ----------------------------------------------------------------------------------------------------------------------------------
% docDynamo PDF template
% ----------------------------------------------------------------------------------------------------------------------------------
\documentclass[letterpaper,12pt]{article}

% Add hyperlinks to TOC
% ----------------------------------------------------------------------------------------------------------------------------------
\usepackage{hyperref}

% Allow EPS files
% ----------------------------------------------------------------------------------------------------------------------------------
\usepackage{graphicx}
\usepackage{epstopdf}

% Use the caption package to enable captions that are not numbered (caption*)
% ----------------------------------------------------------------------------------------------------------------------------------
\usepackage[font=small,textfont=it,justification=justified,singlelinecheck=false]{caption}

% Create a light gray color to use for source code listings
% ----------------------------------------------------------------------------------------------------------------------------------
\usepackage[table]{xcolor}
\definecolor{ltgray}{HTML}{E8E8E8}
\definecolor{dkblue}{HTML}{396A93}

% Styling for hyperlinks
% ----------------------------------------------------------------------------------------------------------------------------------
\hypersetup{frenchlinks=true}
% {
%    colorlinks,
%    linkcolor={dkblue},
%    citecolor={dkblue},
%    urlcolor={dkblue}
%}

% Use listings package instead of verbatim for displaying code
% ----------------------------------------------------------------------------------------------------------------------------------
\usepackage{courier}

\usepackage[warn]{textcomp}
\usepackage{listings}
\lstset
{
    basicstyle=\small\ttfamily,
    columns=flexible,
    breaklines=true,
    frame=tb,
    backgroundcolor=\color{ltgray},
    upquote=true
}

% Use tabularx for tables
% ----------------------------------------------------------------------------------------------------------------------------------
\usepackage{tabularx}
% \usepackage{ltablex}  This cannot be used because there is no package for RHEL/CENTOS
\newcolumntype{R}{>{\raggedleft\arraybackslash}X}%
\renewcommand{\arraystretch}{1.3}

% Allow four section levels (The fourth is implemented with paragraph)
% ----------------------------------------------------------------------------------------------------------------------------------
\usepackage{titlesec}
\setcounter{secnumdepth}{4}

\titleformat{\paragraph}
{\normalfont\normalsize\bfseries}{\theparagraph}{1em}{}
\titlespacing*{\paragraph}{0pt}{3.25ex plus 1ex minus .2ex}{1.5ex plus .2ex}

% Section styling
% ----------------------------------------------------------------------------------------------------------------------------------
\usepackage{sectsty}
\allsectionsfont{\color{dkblue}}

% Define source code highlighting
% ----------------------------------------------------------------------------------------------------------------------------------
\newcommand{\Hilight}{\makebox[0pt][l]{\color{cyan}\rule[-4pt]{0.65\linewidth}{14pt}}}

% Set the font to Helvetica
% ----------------------------------------------------------------------------------------------------------------------------------
\usepackage{helvet}
\renewcommand{\familydefault}{\sfdefault}

% Set margins
% ----------------------------------------------------------------------------------------------------------------------------------
\usepackage[top=.9in, bottom=1in, left=.5in, right=.5in]{geometry}

% Sections start a new page
% ----------------------------------------------------------------------------------------------------------------------------------
\let\stdsection\section
\renewcommand\section{\newpage\stdsection}

% Format paragraphs with no indent and a blank line between paragraphs
% ----------------------------------------------------------------------------------------------------------------------------------
\setlength\parindent{0pt}
\usepackage{parskip}

% Add page headers and footers
% ----------------------------------------------------------------------------------------------------------------------------------
\usepackage{fancyhdr}

\fancyhead[LE,RO]{\slshape \rightmark}
\fancyhead[LO,RE]{\slshape \leftmark}

\fancypagestyle{plain}
{
    \fancyhead{}
    \lhead[]{TABLE OF CONTENTS}
}

\lfoot[]{Crunchy Postgres Operator Installation and Quickstarts Guide\\v2.0}
\cfoot[]{\ \\-\ \thepage\ -}
\rfoot[]{Crunchy Data Solutions, Inc.\\\today}
\pagestyle{fancy}

\renewcommand{\headrulewidth}{0.4pt}
\renewcommand{\footrulewidth}{0.4pt}

% Use framed package for admonition
% ----------------------------------------------------------------------------------------------------------------------------------
\usepackage{framed}

% ----------------------------------------------------------------------------------------------------------------------------------
% Begin document
% ----------------------------------------------------------------------------------------------------------------------------------
\begin{document}

% Create the title page
% ----------------------------------------------------------------------------------------------------------------------------------
\hypersetup{pageanchor=false}

\makeatletter
    \begin{titlepage}
        \begin{center}
            {\large \ }\\[18ex]
            {\huge \bfseries Crunchy Postgres Operator}\\[1ex]
            {\large Installation and Quickstarts Guide}\\[12ex]
            {\large v2.0}\\[12ex]
            \includegraphics[width=6in]{/crunchy-docdynamo-postgres-operator/doc/output/latex/logo}\\[12ex]
            {\large Crunchy Data Solutions, Inc.}\\[1ex]
            {\large \today}
        \end{center}
    \end{titlepage}
\makeatother
\thispagestyle{empty}
\newpage

\hypersetup{pageanchor=true}

% Generate TOC
% ----------------------------------------------------------------------------------------------------------------------------------
\setcounter{tocdepth}{3}
\topskip0in
\thispagestyle{plain}
\renewcommand\contentsname{Table of Contents}
\tableofcontents

% ----------------------------------------------------------------------------------------------------------------------------------
% Content
% ----------------------------------------------------------------------------------------------------------------------------------

% 'Project Structure' Section
% ----------------------------------------------------------------------------------------------------------------------------------
\section{Project Structure}\label{/_project_structure}

To perform an installation of the operator, first create the project structure as follows on your host, here we assume a local directory called \textbf{odev}:
\vspace{.75em}
\begin{lstlisting}
export GOPATH=$HOME/odev
mkdir -p $HOME/odev/src $HOME/odev/bin $HOME/odev/pkg
mkdir -p $GOPATH/src/github.com/crunchydata/
\end{lstlisting}

Next, get a tagged release of the source code:
\vspace{.75em}
\begin{lstlisting}
cd $GOPATH/src/github.com/crunchydata
git clone https://github.com/CrunchyData/postgres-operator.git
cd postgres-operator
git checkout 2.6
\end{lstlisting}

% 'Get Images and Binaries' Section
% ----------------------------------------------------------------------------------------------------------------------------------
\section{Get Images and Binaries}\label{/_get_images_and_binaries}

To pull prebuilt versions from Dockerhub of the \textbf{postgres-operator} containers, specify the image versions, and execute the following Makefile target:
\vspace{.75em}
\begin{lstlisting}
export CO_IMAGE_PREFIX=crunchydata
export CO_IMAGE_TAG=centos7-2.6
make pull
\end{lstlisting}

To pull down the prebuilt \textbf{pgo} binaries, download the \textbf{tar.gz} release file from the following link:

\begin{itemize}
    \item \href{https://github.com/CrunchyData/postgres-operator/releases}{Github Releases} 
    \item extract (e.g. tar xvzf postgres-operator.2.6-rc1.tar.gz) 
\end{itemize}\vspace{.75em}
\begin{lstlisting}
cd $HOME
tar xvzf ./postgres-operator.2.6-rc1.tar.gz
\end{lstlisting}

\begin{itemize}
    \item copy \textbf{pgo} client to somewhere in your path (e.g. cp pgo /usr/local/bin) 
\end{itemize}
% 'Installation Prerequsites' Section
% ----------------------------------------------------------------------------------------------------------------------------------
\section{Installation Prerequsites}\label{/_installation_prerequsites}

To run the operator and the \textbf{pgo} client, you will need the following:

\begin{itemize}
    \item a running Kube cluster 
    \item a kubectl client installed and in your PATH and configured to connect to your Kube cluster (e.g. export KUBECONFIG=/etc/kubernetes/admin.conf) 
    \item a Kube namespace created and set to where you want the operator installed, for this install we assume a namespace of \textbf{demo} has been created 
\end{itemize}\vspace{.75em}
\begin{lstlisting}
kubectl create -f examples/demo-namespace.json
kubectl config set-context $(kubectl config current-context) --namespace=demo
kubectl config view | grep namespace
\end{lstlisting}

% 'Basic Installation' Section
% ----------------------------------------------------------------------------------------------------------------------------------
\section{Basic Installation}\label{/_basic_installation}

The basic installation uses the default operator configuration settings, these settings assume you want to use HostPath storage on your Kube cluster for database persistence. Other persistent options are available but require the Advanced Installation below.

% 'Basic Installation' Section, 'Create HostPath Directory' SubSection
% ----------------------------------------------------------------------------------------------------------------------------------
\subsection{Create HostPath Directory}\label{/_basic_installation/_create_hostpath_directory}

The default Persistent Volume script assumes a default HostPath directory be created called \textbf{/data}:
\vspace{.75em}
\begin{lstlisting}
sudo mkdir /data
sudo chown 777 /data
\end{lstlisting}

Create some sample Persistent Volumes using the following script:
\vspace{.75em}
\begin{lstlisting}
export CO_NAMESPACE=demo
export CO_CMD=kubectl
export COROOT=$GOPATH/src/github.com/crunchydata/postgres-operator
go get github.com/blang/expenv
$COROOT/pv/create-pv.sh
\end{lstlisting}

% 'Basic Installation' Section, 'Deploy the Operator' SubSection
% ----------------------------------------------------------------------------------------------------------------------------------
\subsection{Deploy the Operator}\label{/_basic_installation/_deploy_the_operator}

Next, deploy the operator to your Kube cluster:
\vspace{.75em}
\begin{lstlisting}
cd $COROOT
make deployoperator
\end{lstlisting}

Instead of using the bash script you can also deploy the operator using the provided Helm chart:
\vspace{.75em}
\begin{lstlisting}
cd $COROOT/chart
helm install ./postgres-operator
helm ls
\end{lstlisting}

% 'Basic Installation' Section, 'Verify Operator Status' SubSection
% ----------------------------------------------------------------------------------------------------------------------------------
\subsection{Verify Operator Status}\label{/_basic_installation/_verify_operator_status}

To verify that the operator is deployed and running, run the following:
\vspace{.75em}
\begin{lstlisting}
kubectl get pod --selector=name=postgres-operator
\end{lstlisting}

You should see output similar to this:
\vspace{.75em}
\begin{lstlisting}
NAME                                 READY     STATUS    RESTARTS   AGE
postgres-operator-56598999cd-tbg4w   2/2       Running   0          1m
\end{lstlisting}

There are 2 containers in the operator pod, both should be \textbf{ready} as above.

The operator creates the following Custom Resource Definitions:
\vspace{.75em}
\begin{lstlisting}
kubectl get crd
NAME                             AGE
pgbackups.cr.client-go.k8s.io    2d
pgclusters.cr.client-go.k8s.io   2d
pgingests.cr.client-go.k8s.io    2d
pgpolicies.cr.client-go.k8s.io   2d
pgreplicas.cr.client-go.k8s.io   2d
pgtasks.cr.client-go.k8s.io      2d
pgupgrades.cr.client-go.k8s.io   2d
\end{lstlisting}

At this point, the server side of the operator is deployed and ready.

The complete set of environment variables used in the installation so far are:
\vspace{.75em}
\begin{lstlisting}
export CO_IMAGE_PREFIX=crunchydata
export CO_IMAGE_TAG=centos7-2.6
export GOPATH=$HOME/odev
export GOBIN=$GOPATH/bin
export PATH=$PATH:$GOBIN
export COROOT=$GOPATH/src/github.com/crunchydata/postgres-operator
export CO_CMD=kubectl
\end{lstlisting}

You would normally add these into your \textbf{.bashrc} at this point to be used later on or if you want to redeploy the operator.

% 'Basic Installation' Section, 'Configure \textbf{pgo} Client' SubSection
% ----------------------------------------------------------------------------------------------------------------------------------
\subsection{Configure \textbf{pgo} Client}\label{/_basic_installation/_configure_b_pgo_b_client}

The \textbf{pgo} command line client requires TLS for securing the connection to the operator's REST API. This configuration is performed as follows:
\vspace{.75em}
\begin{lstlisting}
export PGO_CA_CERT=$COROOT/conf/apiserver/server.crt
export PGO_CLIENT_CERT=$COROOT/conf/apiserver/server.crt
export PGO_CLIENT_KEY=$COROOT/conf/apiserver/server.key
\end{lstlisting}

The \textbf{pgo} client uses Basic Authentication to authenticate to the operator REST API, for authentication, add the following \textbf{.pgouser} file to your \$HOME:
\vspace{.75em}
\begin{lstlisting}
echo "username:password" > $HOME/.pgouser
\end{lstlisting}

The \textbf{pgo} client needs the URL to connect to the operator.

Depending on your Kube environment this can be done the following ways:

% 'Basic Installation' Section, 'Configure \textbf{pgo} Client' SubSection, 'Running Kube Locally' SubSubSection
% ----------------------------------------------------------------------------------------------------------------------------------
\subsubsection{Running Kube Locally}\label{/_basic_installation/_configure_b_pgo_b_client/_running_kube_locally}

If your local host is not set up to resolve Kube Service DNS names, you can specify the operator IP address as follows:
\vspace{.75em}
\begin{lstlisting}
kubectl get service postgres-operator
NAME                TYPE       CLUSTER-IP     EXTERNAL-IP   PORT(S)          AGE
postgres-operator   NodePort   10.109.184.8   <none>        8443:30894/TCP   5m

export CO_APISERVER_URL=https://10.109.184.8:8443
pgo version
\end{lstlisting}

You can also define a bash alias like:
\vspace{.75em}
\begin{lstlisting}
alias setip='export CO_APISERVER_URL=https://`kubectl get service postgres-operator -o=jsonpath="{.spec.clusterIP}"`:8443'
\end{lstlisting}

This alias will set the CO\_APISERVER\_URL IP address for you!

% 'Basic Installation' Section, 'Configure \textbf{pgo} Client' SubSection, 'Running Kube Remotely' SubSubSection
% ----------------------------------------------------------------------------------------------------------------------------------
\subsubsection{Running Kube Remotely}\label{/_basic_installation/_configure_b_pgo_b_client/_running_kube_remotely}

Set up a port-forward tunnel from your host to the Kube remote host, specifying the operator pod:
\vspace{.75em}
\begin{lstlisting}
kubectl get pod --selector=name=postgres-operator
NAME                                 READY     STATUS    RESTARTS   AGE
postgres-operator-56598999cd-tbg4w   2/2       Running   0          8m

kubectl port-forward postgres-operator-56598999cd-tbg4w 8443:8443
\end{lstlisting}

In another terminal:
\vspace{.75em}
\begin{lstlisting}
export CO_APISERVER_URL=https://127.0.0.1:8443
pgo version
\end{lstlisting}

% 'Basic Installation' Section, 'Verify \textbf{pgo} Client' SubSection
% ----------------------------------------------------------------------------------------------------------------------------------
\subsection{Verify \textbf{pgo} Client}\label{/_basic_installation/_verify_b_pgo_b_client}

At this point you should be able to connect to the operator as follows:
\vspace{.75em}
\begin{lstlisting}
pgo version
pgo client version 2.6
apiserver version 2.6
\end{lstlisting}

\textbf{pgo} commands are documented on the \href{docs/commands.asciidoc}{Commands} page.

% 'Custom Installation' Section
% ----------------------------------------------------------------------------------------------------------------------------------
\section{Custom Installation}\label{/_custom_installation}

Most users after they try out the operator will want to create a more customized installation and deployment of the operator.

% 'Custom Installation' Section, 'Specify Storage' SubSection
% ----------------------------------------------------------------------------------------------------------------------------------
\subsection{Specify Storage}\label{/_custom_installation/_specify_storage}

The operator will work with HostPath, NFS, and Dynamic Storage.

% 'Custom Installation' Section, 'Specify Storage' SubSection, 'NFS' SubSubSection
% ----------------------------------------------------------------------------------------------------------------------------------
\subsubsection{NFS}\label{/_custom_installation/_specify_storage/_nfs}

To configure the operator to use NFS for storage, a sample \textbf{pgo.yaml.nfs} file is provided. Overlay the default \textbf{pgo.yaml} file with that file:
\vspace{.75em}
\begin{lstlisting}
cp $COROOT/examples/pgo.yaml.nfs $COROOT/conf/apiserver/pgo.yaml
\end{lstlisting}

Edit the \textbf{pgo.yaml} file to specify the NFS GID that is set for the NFS volume mount you will be using, the default value assumed is \textbf{nfsnobody} as the GID (65534). Update the value to meet your NFS security settings.

There is currently no script available to create your NFS Persistent Volumes but you can typically modify the \$COROOT/pv/create-pv.sh script to work with NFS.

% 'Custom Installation' Section, 'Specify Storage' SubSection, 'Dynamic Storage' SubSubSection
% ----------------------------------------------------------------------------------------------------------------------------------
\subsubsection{Dynamic Storage}\label{/_custom_installation/_specify_storage/_dynamic_storage}

To configure the operator to use Dynamic Storage classes for storage, a sample \textbf{pgo.yaml.storageclass} file is provided. Overlay the default \textbf{pgo.yaml} file with that file:
\vspace{.75em}
\begin{lstlisting}
cp $COROOT/examples/pgo.yaml.storageclass $COROOT/conf/apiserver/pgo.yaml
\end{lstlisting}

Edit the \textbf{pgo.yaml} file to specify the storage class you will be using, the default value assumed is \textbf{standard} which is the name used by default within a GKE Kube cluster deployment. Update the value to match your storage classes.

Notice that the \textbf{FsGroup} setting is required for most block storage and is set to the value of \textbf{26} since the PostgreSQL container runs as UID \textbf{26}.

% 'Custom Installation' Section, 'Change the Operator Configuration' SubSection
% ----------------------------------------------------------------------------------------------------------------------------------
\subsection{Change the Operator Configuration}\label{/_custom_installation/_change_the_operator_configuration}

There are many ways to configure the operator, those configurations are documented on the \href{docs/configuration.asciidoc}{Configuration} page.

Reasonable defaults are specified which allow users to typically run the operator at this point so you might not initially require any customization beyond specifying your storage.

% 'Custom Installation' Section, 'Deploy and Run' SubSection
% ----------------------------------------------------------------------------------------------------------------------------------
\subsection{Deploy and Run}\label{/_custom_installation/_deploy_and_run}

At this point, you can use the Basic Installation Deploy steps to deploy the operator and run the \textbf{pgo} client.
% 'Overview' Section
% ----------------------------------------------------------------------------------------------------------------------------------
\section{Overview}\label{/_overview}

There are currently \textbf{quickstart} scripts that seek to automate the deployment to popular Kube environments:

\begin{itemize}
    \item \href{../examples/quickstart-for-gke.sh}{quickstart-for-gke.sh} 
    \item \href{../examples/quickstart-for-ocp.sh}{quickstart-for-ocp.sh} 
\end{itemize}
The \textbf{quickstart-for-gke} script will deploy the operator to a GKE Kube cluster.

The \textbf{quickstart-for-ocp} script will deploy the operator to an Openshift Container Platform cluster.

Both scripts assume you have a StorageClass defined for persistence.

Pre-compiled versions of the Operator \textbf{pgo} client are provided for the x86\_64, Mac OSX, and Windows hosts.

% 'Quickstart' Section
% ----------------------------------------------------------------------------------------------------------------------------------
\section{Quickstart}\label{/_quickstart}

% 'Quickstart' Section, 'GKE/PKS' SubSection
% ----------------------------------------------------------------------------------------------------------------------------------
\subsection{GKE/PKS}\label{/_quickstart/_gke_pks}

The \textbf{quickstart-for-gke.sh} script will allow users to set up the Postgres Operator quickly on GKE including PKS. This script is tested on GKE but can be modified for use with other Kubernetes environments as well.

The script requires a few things in order to work:

\begin{itemize}
    \item wget utility installed 
    \item kubectl utility installed 
    \item StorageClass defined 
\end{itemize}
Executing the script will give you a default Operator deployment that assumes \textbf{dynamic} storage and a storage class named \textbf{standard}, things that GKE provides.

The script performs the following:

\begin{itemize}
    \item downloads the Operator configuration files 
    \item sets the \$HOME/.pgouser file to default settings 
    \item deploys the Operator Deployment 
    \item sets your .bashrc to include the Operator environment variables 
    \item sets your \$HOME/.bash\_completion file to be the \textbf{pgo} bash\_completion file 
\end{itemize}
% 'Quickstart' Section, 'Openshift Container Platform' SubSection
% ----------------------------------------------------------------------------------------------------------------------------------
\subsection{Openshift Container Platform}\label{/_quickstart/_openshift_container_platform}

A similar script for installing the operator on OCP is offered with similar features as the GKE script. This script is tested on OCP 3.7 with a StorageClass defined.

% ----------------------------------------------------------------------------------------------------------------------------------
% End document
% ----------------------------------------------------------------------------------------------------------------------------------
\end{document}
